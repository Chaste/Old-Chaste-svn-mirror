\documentclass{article}

\author{Joe Pitt-Francis}
\title{Parallelization of Heart Practical 6}

\begin{document}
\maketitle
\section{Introduction}

This document concerns the parallization of heart practical 6.  As of
this date the code which runs the practical is contained in {\tt
Chaste/heart/tests/TestMonodomainDg0Assembler.hpp} and consists of tests
of the form {\tt testMonodomainDg0?D} (where ?\ is replaced by the
dimension of the problem).  All these tests have the same basic
structure:
\begin{enumerate}
\item
A mesh is constructed from a mesh reader.
\item 
An {\tt EulerIvpOdeSolver} is constructed.
\item 
A {\tt MonodomainPde} object is constructed using the number of
nodes in the mesh, the Euler solver and some time parameters.  This
object will contain the ODE parameters, ODE solutions and stimulus
functions at each of the node locations.
\item
The initial conditions at each location are set by preparing a std::vector
and using {\tt SetUniversalInitialConditions}.
\item
A stimulus function is set on (at least) one of the nodes.
\item
A boundary condition container is constructed to contain Neumann
boundary at all boundaries.
\item
A {\tt SimpleLinearSolver} (which is a wrapper to a KSP) is
constructed.
\item A {\tt MonoDomainDg0Assemble} object is constructed.
\item A vector of voltages (-84.5  for each node) is built.
\item A data writer is initialized.
\item In a loop over time
\begin{itemize}
\item The voltages are used to set the initial condition of the
assembler.
\item The assembler is used to solve the PDE (by repeatedly assembling
and solving a linear system). This returns a new vector of voltages
which is used in the next iteration.
\item The data writers write some data.
\end{itemize}
\item The final solution may be tested against some known solution.
\end{enumerate}

\section{What can by parallelized?}

\subsection{Linear algebra}

The linear algebra components of the code are already written in
parallel since they use PETSc vectors and matrices.  The core linear
algebra solver (KSP) is aware of the parallel structure of the
matrices and vectors and is written in parallel.  There are many
efficiency savings to be made in terms of sparse matrix storage and
packing the matrix block diagonal, but these won't effect the
correctness of the solver.

A potential problem is in interrogating PETSc vectors in order to test
solutions when the code is run in parallel.  Consider this code (which
is correct when run sequentially:
\begin{verbatim}
PetscScalar *solution_elements;
VecGetArray(solution_vector, &solution_elements);
TS_ASSERT_DELTA(solution_elements[0], 1.0, 0.000001);
TS_ASSERT_DELTA(solution_elements[1], 2.0, 0.000001);
TS_ASSERT_DELTA(solution_elements[2], 3.0, 0.000001);
\end{verbatim}

The problem with this test is that {\tt solution\_vector} will be split
between processors.  When run on two processors, process 0 owns the
first 2 components of the vector and process 1 owns the final
component.  The pointer {\tt solution\_elements} points to the portion
of the vector which is locally owned.  If the above code is run on two
processors, process 0 will fail the final assertion (since it doesn't
own the final element), and process 1 will fail all the assertions.
The correct code is to only tests the locally owned portions of the vector:
\begin{verbatim}
PetscScalar *solution_elements;
VecGetArray(solution_vector, &solution_elements);
int lo, hi;
VecGetOwnershipRange(solution_vector,&lo,&hi);
double real_solution[3]={1.0,2.0,3.0};
for (int i=0;i<3;i++){
  if (lo<=i && i<hi){
    TS_ASSERT_DELTA(solution_elements[i-lo], real_solution[i], 0.000001);
  }
}R
\end{verbatim}

\subsection{Linear system assembly}

Three solutions...

\subsection{ODE solvers}

\section{Fine-grained tasks}

\begin{itemize}
\item[3]
Make separate directory for parallel tests which are always run on
more than one virtual processor.  (Done Tues AM).
\item[1]
Fix some linear algebra tests so that they pass when run in
parallel. (Done Tues AM). 
\item[3] 
Fix linear system class so that assemblers can only add or insert to
values that are held locally. (Done Weds AM).
\item[3]
Make a simple PDE on a mesh work (Done Tues PM for heat equation on a disk).
\item[15]
Make {\tt MonodomainPDE} into a distributed object. The distribution
should mirror the distribution of a PETSc vector.
\item[15]
Make data writers work in parallel.
\item[2] 
Copy and fix the three Heart Practical 6 tests.
\item[42] (Total estimate in pair hours.  32 remaining)
\end{itemize}
\end{document}
